\documentclass[./UsersGuide.tex]{subfiles}
 
\begin{document}

\section{\texttt{UQpy} Modules, Classes, \& Functions}

\texttt{UQpy} is structured in five core modules, each centered around specific functionalities:
\begin{enumerate}
\item \texttt{SampleMethods}: This module contains a set of classes and functions to draw samples from random variables. These samples may be randomly drawn, as in Monte Carlo simulation, or they may be deterministically drawn as in stochastic collocation or quasi-Monte Carlo.
\item \texttt{Inference}: This module contains a set of classes and functions to conduct probabilistic inference. The module contains methods that are based on Bayesian, frequentist, likelihood, and information theories. 
\item \texttt{Reliability}: This module contains a set of classes and functions designed specifically to estimate probability of failure.
\item \texttt{Surrogate}: This module contains a set of classes and functions for building surrogate models, meta-models, or emulators.
\item \texttt{RunModel}: This module contains a set of classes and functions that allows \texttt{UQpy} to initiate simulations using either python or third-party computational solvers.
\end{enumerate}
The following sections detail the classes and functions in each module with reference to examples that illustrate their use. Guidance is based on usage in IDE Model (see Section \ref{sec:IDE_Mode})

\subsection{\texttt{SampleMethods} Module}

The \texttt{SampleMethods} module consists of classes and functions to draw samples from random variables. It is imported in a python script using the following command:

\vspace{4mm}
\texttt{{\color{blue} from} \texttt{UQpy} {\color{blue} import} SampleMethods }
\vspace{4mm}

\noindent
The \texttt{SampleMethods} module has the following classes, each corresponding to a different sampling method:

\vspace{4mm}
\begin{center}
	\begin{tabular}{ |l|l| } 
		\hline
		\textbf{Class} &  \textbf{Description} \\
		\hline
		\texttt{MCS}&  Monte Carlo Sampling  \\ 
		\hline
		\texttt{LHS}&  Latin Hypercube Sampling  \\ 
		\hline
		\texttt{STS}&  Stratified Sampling  \\ 
		\hline
		\texttt{PSS}&  Partially Stratified Sampling  \\ 
		\hline
		\texttt{MCMC}&  Markov Chain Monte Carlo  \\ 
		\hline
		\texttt{SROM}&  Stochastic Reduced Order Model  \\ 
		\hline
	\end{tabular}
\end{center}
\vspace{4mm}

\noindent
Each class can be imported individually into a python script. For example, the MCMC class can be imported to a script using the following command:

\vspace{4mm}
\texttt{{\color{blue} from} \texttt{UQpy.SampleMethods} {\color{blue} import} MCMC}
\vspace{4mm}

\noindent
The following subsections describe each class, their respective inputs and attributes, and their use.

\subsubsection{\texttt{UQpy.SampleMethods.MCS}}

\subsubsection{\texttt{UQpy.SampleMethods.LHS}}

\begin{center}
	\begin{tabular}{ |l|c|l| } 
		\hline
		\textbf{Property} & \textbf{Type} & \textbf{Options} \\
		\hline
		\texttt{\#criterion}& \textit{string} &  'random', 'centered', 'maximin', 'correlate'  \\ 
		\hline
		\multirow{5}{*}{\texttt{\#distance}} & \multirow{5}{*}{\textit{string}} & 'braycurtis', 'canberra', 'chebyshev', 'cosine', \\ 
		&  &  'dice', 'euclidean', 'hamming', 'jaccard',  'cityblock',   \\ 
		&  &  'matching', 'minkowski', 'rogerstanimoto',  'correlation', \\ 
		&  & 'sokalmichener', 'sokalsneath', 'sqeuclidean',  \\ 
		&  & ' 'kulsinski', 'mahalanobis', 'russellrao', 'seuclidean',  \\ 
		\hline
	\end{tabular}
\end{center}

\subsubsection{\texttt{UQpy.SampleMethods.STS}}

\subsubsection{\texttt{UQpy.SampleMethods.PSS}}


\subsubsection{\texttt{UQpy.SampleMethods.MCMC}}

The attributes of the \texttt{MCMC} class are listed below:

\begin{center}
%	\resizebox{\textwidth}{!}{
	\begin{tabular}{ |l|c|c|c| } 
				\hline
		\multicolumn{4}{|c|}{\texttt{MCMC} Class Attributes} \\
		\hline
		\textbf{Attribute} & \textbf{Input/Output} & \textbf{Required} & \textbf{Optional} \\
		\hline
		\texttt{dimension} & Input &  & $\star$  \\ 
		\hline
		\texttt{pdf\_proposal\_type} & Input & & $\star$   \\ 
		\hline
		\texttt{pdf\_proposal\_scale} & Input &  & $\star$  \\ 
		\hline
		\texttt{pdf\_target\_type}& Input &  &  $\star$  \\ 
		\hline
		\texttt{pdf\_target} & Input & $\star$ &   \\ 
		\hline
		\texttt{pdf\_target\_params} & Input  & &  $\star$  \\ 
		\hline
		\texttt{algorithm} & Input &  & $\star$  \\ 
		\hline
		\texttt{jump} & Input &  & $\star$  \\ 
		\hline
		\texttt{nsamples}& Input & $\star$ &    \\ 
		\hline
		\texttt{seed} & Input & & $\star$   \\ 
		\hline
		\texttt{nburn} & Input & & $\star$   \\ 
		\hline
		\texttt{samples} & Output & & \\
		\hline
	\end{tabular}%}
\end{center}

\noindent
A brief description of each attribute can be found in the table below:

\begin{center}
%	\resizebox{\textwidth}{!}{
	\begin{tabular}{ |l|c|c|c| } 
				\hline
		\multicolumn{4}{|c|}{\texttt{MCMC} Class Attributes} \\
		\hline
		\textbf{Attribute} & \textbf{Type} & \textbf{Options} & \textbf{Default} \\
		\hline
		\texttt{dimension} & {\it integer} &  & \texttt{dimension} = 1  \\ 
		\hline
		\texttt{pdf\_proposal\_type} & {\it string} & \begin{tabular}[t]{c}`Normal'\\`Uniform' \end{tabular}& `Uniform'   \\ 
		\hline
		\texttt{pdf\_proposal\_scale} & \begin{tabular}[t]{c}{\it float}\\{\it float list} \end{tabular} &  & \begin{tabular}[t]{c}[1,1,\dots,1]\\len = \texttt{dimension} \end{tabular}  \\ 
		\hline
		\texttt{pdf\_target\_type}& {\it string} & \begin{tabular}[t]{c}`marginal\_pdf'\\`joint\_pdf' \end{tabular} &  `marginal\_pdf'  \\ 
		\hline
		\texttt{pdf\_target} & \begin{tabular}[t]{c}{\it function}\\{\it string} \end{tabular} &  &  \\ 
		\hline
		\texttt{pdf\_target\_params} & Input  & &  $\star$  \\ 
		\hline
		\texttt{algorithm} & Input &  & $\star$  \\ 
		\hline
		\texttt{jump} & Input &  & $\star$  \\ 
		\hline
		\texttt{nsamples}& Input & $\star$ &    \\ 
		\hline
		\texttt{seed} & Input & & $\star$   \\ 
		\hline
		\texttt{nburn} & Input & & $\star$   \\ 
		\hline
		\texttt{samples} & Output & & \\
		\hline
	\end{tabular}%}
\end{center}


\begin{center}
	\begin{tabular}{ |l|c|l| } 
		\hline
		\textbf{Property} & \textbf{Type} & \textbf{Description/Options} \\
		\hline
		\texttt{\#criterion}& \textit{string} &  'random', 'centered', 'maximin', 'correlate'  \\ 
		\hline
		\multirow{5}{*}{\texttt{\#distance}} & \multirow{5}{*}{\textit{string}} & 'braycurtis', 'canberra', 'chebyshev', 'cosine', \\ 
		&  &  'dice', 'euclidean', 'hamming', 'jaccard',  'cityblock',   \\ 
		&  &  'matching', 'minkowski', 'rogerstanimoto',  'correlation', \\ 
		&  & 'sokalmichener', 'sokalsneath', 'sqeuclidean',  \\ 
		&  & ' 'kulsinski', 'mahalanobis', 'russellrao', 'seuclidean',  \\ 
		\hline
	\end{tabular}
\end{center}


\begin{center}
	\begin{tabular}{ |l|l| } 
		\hline
		\textbf{Property} &  \textbf{Options} \\
		\hline
		\texttt{\#target distribution}&   'multivariate\_pdf', 'marginal\_pdf', 'normal\_pdf'  \\ 
		\hline
		\texttt{\#proposal distribution}&   'Uniform', 'Normal'  \\ 
		\hline
		\texttt{\#algorithm}&   'MH', 'MMH'  \\ 
		\hline
	\end{tabular}
\end{center}

\subsubsection{\texttt{UQpy.SampleMethods.SROM}}



\subsubsection{Adding a sampling method in UQpy}


\end{document}